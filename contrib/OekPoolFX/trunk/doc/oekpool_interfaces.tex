\documentclass[	
				a4paper, 
				twoside,
				11pt,
				DIV11,
				BCOR12mm,
				bibtotoc, 
				halfparskip, 
				headsepline, 
				pointlessnumbers]{scrartcl}

\usepackage[utf8]{inputenc}
\usepackage[ngerman]{babel}
\usepackage[T1]{fontenc}
\usepackage[margin=2cm, includehead]{geometry}
\usepackage{bibgerm}
%\usepackage[numbers]{natbib}
\usepackage[dvips]{graphicx}
\usepackage{amsmath}
\usepackage{amssymb}
\usepackage{pstricks}
\usepackage{pst-math}
\usepackage{pstricks-add}
\usepackage{pst-plot}
\usepackage{pst-tree}
\usepackage{rotating}
\usepackage{fancybox}
\usepackage{fancyhdr}
\usepackage{colortbl}
\usepackage{listings}
\usepackage{url}
\usepackage{tikz}
\usepackage{graphics}
%\usepackage{graphicx}
% Define user colors using the RGB model
\definecolor{dunkelgrau}{rgb}{0.8,0.8,0.8}
\definecolor{hellgrau}{rgb}{0.95,0.95,0.95}
\definecolor{rot}{rgb}{1,0,0}
\definecolor{hellrot}{rgb}{1,0.7,0.7}
\definecolor{hellgruen}{rgb}{0.7,1,0.7}
\definecolor{hellgelb}{rgb}{1,1,0.5}

%============================================================================
\title{Schnittstellen für die OekPool-Anwendung}
\author{Julian Bader und Bastian Wißler}
\date{\today}
%============================================================================


\pagestyle{fancy}
\fancyhf{}
\fancyhead[L]{\leftmark}
\fancyhead[R]{\rightmark}
\renewcommand{\headrulewidth}{0.2pt}
\fancyfoot[C]{\thepage}

\makeindex

%==============================================================================================================

\begin{document}

\maketitle
\indent
\begin{abstract}
\section*{Abstract}
In diesem Dokument werden alle externen Schnittstellen von und zur OekPool-Anwendung definiert.
Hierbei werden sowohl die Serverapplikation als auch das Webinterface betrachtet.
Als Dienst zur Speicherung anwendungsrelevanter Daten kommt LDAP zum Einsatz.
Zur Konfiguration der Clients wird TFTP und PXELinux verwendet.
Die Clients an sich werden per SSH gesteuert.
\end{abstract}
\newpage

\section{LDAP}
Im LDAP-Verzeichnis müssen drei Arten von Informationen enthalten sein.
Für Verwaltungs- und Identifikationszwecke benötigt die Anwendung clientbezogene Eigenschaften.
Um eine Zeitsteuerung zu ermöglichen werden für jeden Client Informationen bzgl. assoziierter PXE-Konfigurationen verwendet.
Damit die Wake-on-LAN auch in anderen IP-Netzen möglich ist, sind Informationen über die vom DHCP-Server verwalteten Netzwerke eforderlich.

\subsection{Clientbezogene Eigenschaften}
\subsection{Zeit- und PXE-Konfiguration}
\subsection{DHCP-Informationen}

\section{TFTP}
Auf dem TFTP-Server befinden sich, der Client-Konfiguration entsprechend, zwei Ausprägungen von PXE-Menüs.
Für alle Clients, die nicht über die Anwendung verwaltet werden bzw. nicht immer Zeitslots zugewiesen bekommen, muss ein Default-Menü vorhanden sein.
Sämtliche Menüs, die im LDAP definiert sind müssen einmal auf dem TFTP-Server abgelegt werden, damit sie bei Bedarf (überlagernde Steuerung) verlinkt werden können.

\subsection{Default-Menü}

\subsection{Spezifische PXE-Menüs}

\section{SSH}

\end{document}
